% !TeX root = ./main.tex

\ustcsetup{
  title              = {基于RGB-D相机的实时三维重建及语义分割系统的设计与实现},
  title*             = {Design and Implementation of a Real-time 3D Reconstruction and Semantic Segmentation System Based on RGB-D Camera},
  author             = {贾柠宇},
  author*            = {Jia Ningyu},
  speciality         = {软件工程},
  speciality*        = {Software Engineering},
  supervisor         = {万寿红~副教授},
  supervisor*        = {A.P. Wan Shouhong},
  advisor            = {李嘉茂~研究员},
  advisor*           = {Researcher Li Jiamao},
  % date               = {2017-05-01},  % 默认为今日
  % professional-type  = {专业学位类型},
  % professional-type* = {Professional degree type},
  % department         = {软件学院},  % 院系,本科生需要填写
  % student-id         = {SA21225229},  % 学号,本科生需要填写
  % secret-level       = {秘密},     % 绝密|机密|秘密|控阅,注释本行则公开
  % secret-level*      = {Secret},  % Top secret | Highly secret | Secret
  % secret-year        = {10},      % 保密/控阅期限
  reviewer = true, % 声明页显示“评审专家签名”
  %
  % 数学字体
  % math-style         = GB,  % 可选:GB, TeX, ISO
  math-font = xits, }% 可选:stix, xits, libertinus

% 加载宏包
\usepackage{graphicx}
\usepackage{subfigure}

% 定理类环境宏包
\usepackage{amsthm}

% 插图
\usepackage{graphicx}

% 三线表
\usepackage{booktabs}

% 跨页表格
\usepackage{longtable}

% 算法
\usepackage[ruled,linesnumbered]{algorithm2e}

% SI 量和单位
\usepackage{siunitx}

% 参考文献使用 BibTeX + natbib 宏包
% 顺序编码制
\usepackage[sort]{natbib}
\bibliographystyle{ustcthesis-numerical}

% 著者-出版年制
% \usepackage{natbib}
% \bibliographystyle{ustcthesis-authoryear}

% 本科生参考文献的著录格式
% \usepackage[sort]{natbib}
% \bibliographystyle{ustcthesis-bachelor}

% 参考文献使用 BibLaTeX 宏包
% \usepackage[style=ustcthesis-numeric]{biblatex}
% \usepackage[bibstyle=ustcthesis-numeric,citestyle=ustcthesis-inline]{biblatex}
% \usepackage[style=ustcthesis-authoryear]{biblatex}
% \usepackage[style=ustcthesis-bachelor]{biblatex}
% 声明 BibLaTeX 的数据库
% \addbibresource{bib/ustc.bib}

% 配置图片的默认目录
\graphicspath{{figures/}}

% 数学命令
\makeatletter
\newcommand\dif{%  % 微分符号
  \mathop{}\!%
  \ifustc@math@style@TeX
    d%
  \else
    \mathrm{d}%
  \fi
}
\makeatother
\newcommand\eu{{\symup{e}}}
\newcommand\iu{{\symup{i}}}

% 用于写文档的命令
\DeclareRobustCommand\cs[1]{\texttt{\char`\\#1}}
\DeclareRobustCommand\env[1]{\texttt{#1}}
\DeclareRobustCommand\pkg[1]{\textsf{#1}}
\DeclareRobustCommand\file[1]{\nolinkurl{#1}}

\usepackage{hyperref}

% \renewcommand{\arraystretch}{1.3}

\usepackage{amsmath}

\usepackage{enumitem}
\setlist[enumerate]{label=(\arabic*), wide=\parindent, itemindent=2em}

\usepackage{float}
\usepackage{ctex}

\let\cleardoublepage\clearpage

\SetKwInput{KwData}{Input}
\SetKwInput{KwResult}{Output}