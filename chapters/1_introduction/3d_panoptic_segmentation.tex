\subsection{三维语义分割}

\par 三维语义分割是三维重建行业一个新的研究方向,其目标是将一个三维模型分割成不同的语义区域,通常涉及到两个步骤:首先,从一组图像中重建出三维场景;然后,应用一种分割算法,对每个三维点或体素进行语义标注。
目前,三维语义分割在计算机视觉和深度学习领域已经得到了广泛的研究。\cite{3DSemanticParsing,parislille3d}

\par 卷积神经网络(Convolutional Neural Network,CNN)和图神经网络(Graph Neural Network,GNN)已经在三维语义分割中得到了广泛的应用。例如,PointNet 和其后续版本 PointNet++\cite{pointnet,pointnet++} 使用 CNN 直接处理无序的点云数据,实现了有效的三维分割。
PointCNN\cite{PointCNN} 通过学习$\chi$变换,将无序的点云数据转换成带权重和潜在规范顺序的形式,从而在点云中实现类似传统 CNN 网络卷积操作的特征学习。
此外,一些工作使用 GNN 来处理三维空间的结构信息,以提升分割效果。

\par 为了进一步改善三维重建和语义分割的效果,近年来研究者提出了一些新的技术和方法。例如,PanopticFusion\cite{panopticfusion}首次提出实时地将全景分割图像整合到三维模型中。该方法通过贪心匹配算法将点云与图像中分割出的片段进行匹配,然后整合到TSDF点云模型中\cite{VolumetricMethod,voxblox},并且通过加权函数调整每个点对应的语义类别标签。其中,语义类别标签根据语义分割图像提供的置信度分数生成。此外,他们还应用了条件随机场(Conditional Random Field,CRF)来规范重建过程,从而提高了重建精度。在 PanopticFusion 的基础上,Voxblox 对深度图像进行分割来提高语义分割的准确性,使用几何分割结果和语义分割结果计算出成对权重(Pair-Wise Weight),用于点云类别标签的匹配。
 
\par 一些研究者基于 Voxblox 和 Voxblox++\cite{voxblox++},对三维重建语义融合的过程作出两个改进\cite{ReconstructingInteractive3D}:将类别标签加入到成对权重中,组合成三元权重(Triplet Weight);另外,如果语义分割图像中两个片段的重叠度达到一定阈值,则它们也会被融合。同时,他们还介绍了一种基于几何分割、语义分割和物理定律的方法,将CAD模型与重建物体进行对齐。该方法可以重建出接触图和交互式虚拟场景,使机器人与环境互动。

\par 在实时性方面,Online SegFusion\cite{OnlineSegFusion} 提出了一个基于深度神经网络的实时视觉框架,可以在重建出室内场景三维模型的同时进行语义融合更新。该方法放弃了在线深度融合中使用路由网络,使用涡流池化块(Vortex Pooling)保留高频表面细节,可以有效地融合语义标签和三维模型。

\par 尽管以上研究已经取得了显著的成果,但是在实时三维重建和语义分割领域,目前却仍然面临一些挑战。首先是实时处理大规模数据的问题。由于三维数据,尤其是点云数据,通常规模较大,因此处理这些数据需要大量的计算资源。虽然一些优化算法和多进程加速技术可以提高处理速度\cite{VoxelHashing,Real-TimeVolumetric,Multi-resolutionSurfaceReconstruction},但如何在保持高精度的同时实现实时处理仍然是一个挑战,需要使用更高级的算法和硬件资源(如 GPU)\cite{GPUAccelerated}。

\par 其次,虽然神经网络已经在三维分割中取得了很好的性能,但是如何有效地将语义信息融入到三维模型中仍然是一个难题。一方面,分割模型可能会产生错误或不确定的标签;另一方面,语义信息和几何信息可能会存在不一致性。解决这一问题需要更高级的语义融合算法。

\par 最后,动态环境的处理也是一个难题。环境中发生位移的物体会引入噪声和不一致性,影响三维重建和语义分割的精度\cite{VideoPanopticSegmentation}。解决这个问题需要系统能够增量式更新三维模型,这在技术上具有非常大的挑战性。

\par 这些挑战需要在算法、模型和硬件资源等多个方面进行创新和优化,以实现更好的实时三维重建和语义分割。

% 三维语义分割是三维重建行业一个新的研究方向,其目标是将一个三维模型分割成不同的语义区域,通常涉及到两个步骤:首先,从一组图像中重建出三 维场景;然后,应用一种分割算法,对每个三维点或体素进行语义标注。 目前三维语义分割在计算机视觉和深度学习领域已经得到了广泛的研究,它的研究现状和发展趋势具有以下特点: 

% (1)深度学习的广泛应用
% 深度学习,特别是卷积神经网络(convolutional neural network,CNN)和图神经网络(graph neural network,GNN),已经在三维语义分割中得到了广泛的应用。例如,PointNet 和其后续版本 PointNet++ 使用 CNN 直接处理无序的点云数据,实现了有效的三维分割。此外,一些工作使用 GNN 来处理三维空间的结构信息,以提升分割性能。 

% (2)全景数据的利用
% 为了捕获场景的全局信息,一些方法开始直接使用全景数据进行三维分割。 例如,一些工作将全景图像转换为球形投影或立方体投影,然后应用二维的深度学习方法进行分割。此外,一些工作结合全景视角和多视图技术,以提高重建和分割的精度。

% (3)大规模数据集的构建
% 大规模的标注数据集对于训练深度学习模型至关重要。例如,ScanNet 和 Matterport3D等数据集提供了大量的室内场景的三维扫描和语义标注, 推动了三维分割的研究。然而,构建这样的数据集需要大量的手工标注, 因此自动或半自动的标注方法是一个重要的研究方向。

% 但是,在实时三维重建和语义分割方面,三维语义分割正在面临一些挑战:

% 首先是实时处理大规模数据的问题。由于三维数据,尤其是点云数据,通常规模较大,因此处理这些数据需要大量的计算资源。虽然一些优化算法和多进程加速技术可以提高处理速度,但如何在保持高精度的同时实现实时处理仍然是一个挑战,需要使用更高级的算法和硬件资源(如 GPU)。

% 其次,尽管深度学习已经在三维分割中取得了很好的性能,但是如何有效地将语义信息融入到三维模型中仍然是一个难题。一方面,分割模型可能会产生错误或不确定的标签;另一方面,语义信息和几何信息可能会存在不一致性。

% 最后,动态环境的处理也是一个难题。环境中发生位移的物体会引入噪声和不一致性,影响三维重建和语义分割的精度。解决这个问题需要系统能够增量式更新三维模型,这在技术上具有非常大的挑战性。

% 这些挑战需要在算法、模型、和硬件资源等多个方面进行创新和优化,以实现更好的实时三维重建和语义分割。 