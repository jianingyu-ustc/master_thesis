\subsection{三维重建}

\par 三维重建是一种将二维图像或者其它类型的数据转换为三维模型的技术。这个过程通常涉及到多个步骤和多种算法,
包括图像处理、特征检测、几何计算等。三维重建在许多领域都有应用,如医学成像、电影制作、虚拟现实和地理信息系统。
例如,在医学领域,可以从一系列的二维计算机断层扫描(Computerized Tomograph,CT)或磁共振成像(Magnetic Resonance Imaging,MRI)扫描图像中重建出三维的人体器官模型,
帮助医生更好地理解患者的病情。\cite{bushong2003magnetic,song2017review}在电影制作和游戏开发中,通过三维扫描和重建技术,可以把真实世界的人物或场景转换为三维模型,用于创造更逼真的视觉效果\cite{SLAMCast,hasenfratz2003survey,finance2015visual}。

\par 基于RGB-D相机的三维重建技术主要利用RGB图像信息和深度图像信息创建三维模型。
这种方法在处理复杂的场景时,比如有遮挡、自反射、各种材质和颜色的场景时,会有更好的效果\cite{SLAMCast,cavallari2019real,Dynamicfusion,Fusion4d},因为 RGB-D 相机提供的深度信息,可以直接用于计算物体的形状和位置,而不需要像传统的立体视觉那样进行复杂的深度计算。
在重建过程中,首先相机会捕捉一系列的 RGB-D 图像,然后通过“配准”的过程,将这些图像对齐在一起,创建一个一致的三维表示。
这个过程需要处理相机移动和旋转带来的视角变化。最后,利用“融合”的技术,将所有深度图像融合在一起,创建一个完整的三维模型。
这个过程需要处理噪声和不一致性,以保证三维模型的质量和准确性。

% \par 基于RGB-D相机的三维重建技术利用RGB图像和捕捉颜色信息,
% 利用深度图像技术物体表面到相机的距离,从而生成三维点云数据。
% 通过将这两种信息融合,可以精确地重建物体的几何形状和表面纹理。

\par 1996年,Curless 和 Levoy 首先提出了基于体素的重建方法\cite{VolumetricMethod},
通过使用一个规则网格来存储表示模型的离散化的符号距离函数(Signed Distance Function,SDF)。
该方法被Rusinkiewicz等人的首个实时重建算法\cite{rusinkiewicz2002real}和后来的
KinectFusion\cite{newcombe2011kinectfusion,izadi2011kinectfusion}采用。
基于体素的重建方法使用 SDF 隐式地存储模型表面,
即内部和外部体素分别存储到最近表面点的负距离和正距离,表面本身被定义为 SDF 的零交叉点,
在执行算法之前,必须定义体素的大小和网格的空间范围。
此外,每个体素中可以存储额外的属性信息,如颜色信息。

\par 常规的体素网格存储效率非常低,在实时场景重建时,大多数方法严重依赖 GPU 的处理能力,并且重建空间范围和分辨率通常受到 GPU 显存的限制。
为了支持更大的空间范围,不同学者已经提出了各种方法来提高基于体素的表示的存储效率\cite{whelan2016elasticfusion,vicini2021non,hornung2013octomap}。
为了防止在当前重建范围之外获取的深度图像而导致数据丢失,Whelan 等人
提出了一种简单的动态移动体素网格的方法\cite{whelan2012kintinuous},使其跟随相机的运动而运动。
该方法将移动到当前重建范围之外的部分体素转换为表面网格并单独存储。
虽然该方法可以实现更大的重建范围,但需要大量的离线内存使用,并且已经无法实时访问已重建完成并流出的表面。